\frontmatter 
% title page, list of tables, list of figures
% T I T L E   P A G E
% -------------------
% This file goes along with the master LaTeX file uw-ethesis.tex
% Last updated May 27, 2009 by Stephen Carr, IST Client Services
% The title page is counted as page `i' but we need to suppress the
% page number.  We also don't want any headers or footers.
\pagestyle{empty}
\pagenumbering{roman}

% The contents of the title page are specified in the "titlepage"
% environment.
\begin{titlepage}
        \begin{center}
        \vspace*{1.0cm}

        \Huge
        {\bf SE 350 Laboratory Project Manual \\
             A Real-time Executive for \\
                 Keil MCB1700}

        \vspace*{1.0cm}

        \normalsize
        by \\

        \vspace*{1.0cm}

        \Large
        Yiqing Huang \\
        Thomas Reidemeister

        \vspace*{3.0cm}

        \normalsize
        Electrical and Computer Engineering Department \\
        University of Waterloo \\ 

        \vspace*{2.0cm}
\makeatletter
        Waterloo, Ontario, Canada, \@date \\
\makeatother
        
      
        
        \vspace*{1.0cm}

        \copyright\ Y. Huang, and T. Reidemeister 2012-2023 \\
        \end{center}
\end{titlepage}

% The rest of the front pages should contain no headers and be numbered using Roman numerals starting with `ii'
\pagestyle{plain}
\setcounter{page}{2}

\cleardoublepage % Ends the current page and causes all figures and tables that have so far appeared in the input to be printed.
% In a two-sided printing style, it also makes the next page a right-hand (odd-numbered) page, producing a blank page if necessary.
%\newpage

\cleardoublepage
%\newpage

% T A B L E   O F   C O N T E N T S
% ---------------------------------
\tableofcontents
\cleardoublepage
%\newpage

% L I S T   O F   T A B L E S
% ---------------------------
\listoftables
\addcontentsline{toc}{chapter}{List of Tables}
\cleardoublepage
%\newpage

% L I S T   O F   F I G U R E S
% -----------------------------
\listoffigures
\addcontentsline{toc}{chapter}{List of Figures}
\cleardoublepage
%\newpage

% L I S T   O F   S Y M B O L S
% -----------------------------
% \renewcommand{\nomname}{Nomenclature}
% \addcontentsline{toc}{chapter}{\textbf{Nomenclature}}
% \printglossary
% \cleardoublepage
% \newpage

% Change page numbering back to Arabic numerals
\pagenumbering{arabic}



%Brief overview of the document, acknowledgement, disclaimer
\chapter{Preface}

The University of Waterloo Software Engineering (SE) SE350 course laboratory project is 
to design a small real-time executive (RTX) and implement it 
on a Keil MCB1700 board populated with an NXP LPC1768 microcontroller.

The main purpose of this document is a quick reference guide of the relevant hardware environment 
and software development tools of the Keil MCB1700 board for completing the laboratory project.
To make the manual self-contained, we also include the project description
\footnote{The original project description was written by Professor Paul Dasiewicz. 
The project description included in this manual is a modified version of the original one.}
% and some notes on RTX design issues 
to further guide students.

There are three parts of the document.
\begin{itemize}
\item {Part I  Lab Project Administration Policy}
\item {Part II  RTX Project Description}
\item {Part III  Design and Implementation Notes}
\item {Part IV  Computing Environment and Keil MCB1700 Development Quick Reference Guide}

\end{itemize}

% A C K N O W L E D G E M E N T S
% -------------------------------

\section*{Acknowledgments}
We would like to sincerely thank Professor Paul Dasiewicz who originally designed the RTX course project and provided us with detailed notes and sample code. We also own many thanks to our students who did this course project in the past and provided constructive feedback. Professor Sebastian Fischmeister made the Keil Boards and MDK-ARM donations possible. Professor Jim Barby provided timely departmental resource towards the development of the course project, without which this project will not be possible to start. Roger Sanderson provided us with all necessary experiment tools and resources, which we are grateful for.  We appreciate that Bernie Roehl has shared his valuable Keil board experiences with us. Our gratitude also goes out to Eric Praetzel who sets up the RTOS lab and also maintains the Keil software on Nexus machines; Laura Winger who managed to customize the boards so that we have the neat plastic cover to protect our hardware. Bob Boy from ARM always answers our questions in a detailed and timely manner. Rollen S. D'Souza shared his recent lab teaching experiences with us and this helped us to improve the manual. Thank everyone who has helped.

% LocalWords:  APIs Keil mqueue ipcs multi gettimeofday README docx ecelinux
% LocalWords:  startup
